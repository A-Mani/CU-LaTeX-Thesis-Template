\RequirePackage{amsmath}
\documentclass[a4paper,11pt,twoside,chapterprefix=TRUE]{scrbook}
%\documentclass[a4paper,11pt,oneside,chapterprefix=TRUE]{scrbook}
\usepackage[utf8]{inputenc}
\usepackage{scrlayer-scrpage}
\usepackage[pass,a4paper]{geometry}
%\usepackage[nointegrals]{wasyssym}
\usepackage[T1]{fontenc}
%\usepackage{lmodern}

\usepackage{latexsym}
\usepackage{amssymb}
\usepackage{amsthm}

\usepackage{url}
\usepackage{cite}
\usepackage{hyperref}

\usepackage{graphicx}
\usepackage{tabularx}
\usepackage{array}
\usepackage[inline]{asymptote}
\usepackage{eqparbox}
%\usepackage{subfig}

\usepackage{mdwmath}
\usepackage{mdwtab}
\usepackage{euler}

%\usepackage{proof}
\usepackage[cmyk]{xcolor}
\usepackage{stmaryrd}
\usepackage{tikz}
\usepackage{enumerate}
\usepackage{titletoc}
\usepackage{makeidx}         % allows index generation
\usepackage{graphicx}        % standard LaTeX graphics tool
                            % when including figure files
\usepackage{multicol}        % used for the two-column index
\usepackage[bottom]{footmisc}% places footnotes at page bottom
\usepackage[none]{ubuntu}
%\fontUbuntuLight, \fontUbuntuMedium or \fontUbuntuCondensed \fontUbuntu
%\usepackage{calrsfs}
%standard
%\usepackage{trajan}

%Remark: Do explore other fonts.

%Other interesting packages

%\usepackage{asymptote}
%\usepackage{algorithm2e}
%\usepackage{}
%\usepackage{}
\newenvironment{calligraphic}{\usefont{T1}{pzc}{m}{it}}{}

% Add your abbreviations here
% Example
\newcommand{\oc}{\mathbf{O}}

\newtheorem{theorem}{Theorem}[chapter]
\newtheorem{definition}{Definition}[chapter]
\newtheorem{proposition}{Prop}[chapter]
\newtheorem{lemma}{Lemma}[chapter]
\newtheorem{remark}{Remark}[chapter]

\renewcommand{\labelitemi}{$\bullet$}

\newcommand{\vect}[2]{(#1_1,#1_2,\dotsc,#1_#2)}
%e.g $vect{x}{n}$

\newcounter{fem}
\newtheorem{examp}[lemma]{Example}
\setcounter{fem}{0}

\makeindex


\begin{document}

\frontmatter


\KOMAoptions{open=left}

\thispagestyle{empty}
\mbox{}
\clearpage
\setcounter{page}{1}
%%%%%%%%%%%%%%%%%%%%%%%%%%%%%%%%%%%%%%%%%%%%%%%%%%%%%%%%%%%%%%%%%%%%%%%%%%%%%%%%%%
% Copyright: Prof(Miss) A. Mani, 2014-                                           %
% Contact: a.mani.cms@gmail.com  Web: http://www.logicamani.in                   %
% This template is licensed under GNU-GPL Version 3 and above                    %
%    This program is free software: you can redistribute it and/or modify
%    it under the terms of the GNU General Public License as published by
%    the Free Software Foundation, either version 3 of the License, or
%    (at your option) any later version.
%
%   This program is distributed in the hope that it will be useful,
%   but WITHOUT ANY WARRANTY; without even the implied warranty of
%   MERCHANTABILITY or FITNESS FOR A PARTICULAR PURPOSE.  See the
%   GNU General Public License for more details.
%
%   You should have received a copy of the GNU General Public License
%   along with this program.  If not, see <http://www.gnu.org/licenses/>.
%
%%%%%%%%%%%%%%%%%%%%%%%%%%%%%%%%%%%%%%%%%%%%%%%%%%%%%%%%%%%%%%%%%%%%%%%%%%%%%%%%%%%
\begin{titlepage}
\begin{center}

% Upper part of the page. The '~' is needed because \\
% only works if a paragraph has started.
%\includegraphics[width=0.15\textwidth]{./logo}~\\[1cm]

%\textsc{\huge University of Calcutta}~\\[1.5cm]

% Title
\hrulefill 

\vspace{5mm}

{ \huge \bfseries \fontUbuntu [Title Here: Template for D.Sc, Ph.D, M.Phil Theses of University of Calcutta  ~\\[0.3cm] }

\hrulefill 

\vspace{36mm}

\begin{center}
{\Large \bfseries \fontUbuntu Thesis submitted for the Degree of $\ldots \ldots$ of the University of Calcutta.} 
\end{center}

\vspace{36mm}

\begin{center}
{\huge \bfseries \fontUbuntu YOUR_NAME}

{\large \bfseries \fontUbuntu Department of Pure Mathematics

University of Calcutta

ADDRESS_LINE

ADDRESS_LINE

Email: {xxxxxxxxxx@xxxx}

Web Page: $\text{http://www.xxxxxxxxxxx xx}$ }

\end{center}

\vspace{18mm}

\hrulefill

\vspace{5mm}
{ \Large \bfseries \fontUbuntu University of Calcutta}

\hrulefill
%{\large \bfseries \fontUbuntu Supervisor: 

%Professor XXXXXXXX XXXXXXXX

%Department of XXXXX

%I.S.I Kolkata}

\end{center}
\end{titlepage}

\newpage
\thispagestyle{empty}
\mbox{}
\clearpage


\newpage
\thispagestyle{empty}

\vspace*{180pt}
\bigskip

\bigskip

\bigskip


\begin{center}
\begin{huge}
%\begin{trjnfamily}
%\begin{sffamily}
\begin{slshape}
%\begin{ttfamily}
%\begin{calligraphic}
{\fontUbuntu \slshape

Poetry,

\bigskip

Poetry 
Poetry,

\bigskip

Poetry

\bigskip

and

\bigskip

Poetry. }
%\end{calligraphic}
%\end{ttfamily}
\end{slshape}
%\end{sffamily}
%\end{trjnfamily}
\end{huge}
\end{center}

\newpage
\thispagestyle{empty}
\mbox{}
\clearpage

\newpage

\vspace{30pt}

\thispagestyle{empty}
\begin{center}
{\huge \bfseries \fontUbuntu Acknowledgement}  
\end{center}

\vspace{30pt}

{\large \fontUbuntu I would like to thank $\ldots$. }

\vspace{15pt}

{\large \fontUbuntu My $\ldots$ and would also like to thank $\ldots$.}

\vspace{15pt}

{\large \fontUbuntu I would also like to thank $\ldots$!}


\tableofcontents

\listoftables
\listoffigures

%\KOMAoptions{open=right}
\mainmatter

\chapter{Introduction}


This template is meant for two sided printing. For one sided printing, you can use 
\begin{verbatim*}
\documentclass[a4paper,11pt,oneside,chapterprefix=TRUE]{scrbook} 
\end{verbatim*}

Note that 
\begin{itemize}
\item {the cover must be printed separately using the \emph{bookcover} package for example.}
\item {the front page does not include fields for supervisor's name etc. These can be added by adjusting the lengths. }
\item {You need to install Ubuntu fonts for \LaTeX or change the font aspects. }
\end{itemize}


%\tableofcontents

\chapter{Notation and Terminology}





\section{Example Definition with Indexing}\label{palg}

\begin{definition}
A \emph{Partial Algebra} $P$ is a tuple of the form \[\left\langle\underline{P},\,f_{1},\,f_{2},\,\ldots ,\, f_{n}, (r_{1},\,\ldots ,\,r_{n} )\right\rangle\] with $\underline{P}$ being a set, $f_{i}$'s being partial function symbols of arity $r_{i}$. The interpretation of $f_{i}$ on the set $\underline{P}$ should be denoted by $f_{i}^{\underline{P}}$, but the superscript will be dropped in this paper as the application contexts are simple enough. If predicate symbols enter into the signature, then $P$ is termed a \emph{Partial Algebraic System}.  \index{Algebra!Partial} 
\end{definition}


\section{Example Proposition}\label{propos}

\begin{proposition}
 \[ 1 + 1 = 2 \text{ or } 1+ 1 \neq 2 \]
\end{proposition}

\begin{proof}
If $1+ 1 =2 $ follows from axioms, 

then \[ 1 + 1 = 2 \text{ or } 1+ 1 \neq 2 \]

\begin{align}
\text{If } 1+ 1 \neq 2\\
\text{then } 1 + 1 = 2 \text{ or } 1+ 1 \neq 2.
\end{align}

\qed
\end{proof}

\section{Extra Files}

Chapters can be kept in separate files and called as below.

\textsf{chapter3.tex} is called by the following command. 

Note that the file should not have any preamble in it.

\include{chapter3}

\include{chapter4}

\include{chapter5}

%\chapter{References}
\backmatter
\bibliographystyle{plain}
\bibliography{bibliofile.bib}

\printindex

\end{document}          








